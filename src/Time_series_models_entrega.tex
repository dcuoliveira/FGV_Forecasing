\documentclass[]{article}
\usepackage{lmodern}
\usepackage{amssymb,amsmath}
\usepackage{ifxetex,ifluatex}
\usepackage{fixltx2e} % provides \textsubscript
\ifnum 0\ifxetex 1\fi\ifluatex 1\fi=0 % if pdftex
  \usepackage[T1]{fontenc}
  \usepackage[utf8]{inputenc}
\else % if luatex or xelatex
  \ifxetex
    \usepackage{mathspec}
  \else
    \usepackage{fontspec}
  \fi
  \defaultfontfeatures{Ligatures=TeX,Scale=MatchLowercase}
\fi
% use upquote if available, for straight quotes in verbatim environments
\IfFileExists{upquote.sty}{\usepackage{upquote}}{}
% use microtype if available
\IfFileExists{microtype.sty}{%
\usepackage{microtype}
\UseMicrotypeSet[protrusion]{basicmath} % disable protrusion for tt fonts
}{}
\usepackage[margin=1in]{geometry}
\usepackage{hyperref}
\hypersetup{unicode=true,
            pdftitle={R Notebook},
            pdfborder={0 0 0},
            breaklinks=true}
\urlstyle{same}  % don't use monospace font for urls
\usepackage{color}
\usepackage{fancyvrb}
\newcommand{\VerbBar}{|}
\newcommand{\VERB}{\Verb[commandchars=\\\{\}]}
\DefineVerbatimEnvironment{Highlighting}{Verbatim}{commandchars=\\\{\}}
% Add ',fontsize=\small' for more characters per line
\usepackage{framed}
\definecolor{shadecolor}{RGB}{248,248,248}
\newenvironment{Shaded}{\begin{snugshade}}{\end{snugshade}}
\newcommand{\AlertTok}[1]{\textcolor[rgb]{0.94,0.16,0.16}{#1}}
\newcommand{\AnnotationTok}[1]{\textcolor[rgb]{0.56,0.35,0.01}{\textbf{\textit{#1}}}}
\newcommand{\AttributeTok}[1]{\textcolor[rgb]{0.77,0.63,0.00}{#1}}
\newcommand{\BaseNTok}[1]{\textcolor[rgb]{0.00,0.00,0.81}{#1}}
\newcommand{\BuiltInTok}[1]{#1}
\newcommand{\CharTok}[1]{\textcolor[rgb]{0.31,0.60,0.02}{#1}}
\newcommand{\CommentTok}[1]{\textcolor[rgb]{0.56,0.35,0.01}{\textit{#1}}}
\newcommand{\CommentVarTok}[1]{\textcolor[rgb]{0.56,0.35,0.01}{\textbf{\textit{#1}}}}
\newcommand{\ConstantTok}[1]{\textcolor[rgb]{0.00,0.00,0.00}{#1}}
\newcommand{\ControlFlowTok}[1]{\textcolor[rgb]{0.13,0.29,0.53}{\textbf{#1}}}
\newcommand{\DataTypeTok}[1]{\textcolor[rgb]{0.13,0.29,0.53}{#1}}
\newcommand{\DecValTok}[1]{\textcolor[rgb]{0.00,0.00,0.81}{#1}}
\newcommand{\DocumentationTok}[1]{\textcolor[rgb]{0.56,0.35,0.01}{\textbf{\textit{#1}}}}
\newcommand{\ErrorTok}[1]{\textcolor[rgb]{0.64,0.00,0.00}{\textbf{#1}}}
\newcommand{\ExtensionTok}[1]{#1}
\newcommand{\FloatTok}[1]{\textcolor[rgb]{0.00,0.00,0.81}{#1}}
\newcommand{\FunctionTok}[1]{\textcolor[rgb]{0.00,0.00,0.00}{#1}}
\newcommand{\ImportTok}[1]{#1}
\newcommand{\InformationTok}[1]{\textcolor[rgb]{0.56,0.35,0.01}{\textbf{\textit{#1}}}}
\newcommand{\KeywordTok}[1]{\textcolor[rgb]{0.13,0.29,0.53}{\textbf{#1}}}
\newcommand{\NormalTok}[1]{#1}
\newcommand{\OperatorTok}[1]{\textcolor[rgb]{0.81,0.36,0.00}{\textbf{#1}}}
\newcommand{\OtherTok}[1]{\textcolor[rgb]{0.56,0.35,0.01}{#1}}
\newcommand{\PreprocessorTok}[1]{\textcolor[rgb]{0.56,0.35,0.01}{\textit{#1}}}
\newcommand{\RegionMarkerTok}[1]{#1}
\newcommand{\SpecialCharTok}[1]{\textcolor[rgb]{0.00,0.00,0.00}{#1}}
\newcommand{\SpecialStringTok}[1]{\textcolor[rgb]{0.31,0.60,0.02}{#1}}
\newcommand{\StringTok}[1]{\textcolor[rgb]{0.31,0.60,0.02}{#1}}
\newcommand{\VariableTok}[1]{\textcolor[rgb]{0.00,0.00,0.00}{#1}}
\newcommand{\VerbatimStringTok}[1]{\textcolor[rgb]{0.31,0.60,0.02}{#1}}
\newcommand{\WarningTok}[1]{\textcolor[rgb]{0.56,0.35,0.01}{\textbf{\textit{#1}}}}
\usepackage{graphicx,grffile}
\makeatletter
\def\maxwidth{\ifdim\Gin@nat@width>\linewidth\linewidth\else\Gin@nat@width\fi}
\def\maxheight{\ifdim\Gin@nat@height>\textheight\textheight\else\Gin@nat@height\fi}
\makeatother
% Scale images if necessary, so that they will not overflow the page
% margins by default, and it is still possible to overwrite the defaults
% using explicit options in \includegraphics[width, height, ...]{}
\setkeys{Gin}{width=\maxwidth,height=\maxheight,keepaspectratio}
\IfFileExists{parskip.sty}{%
\usepackage{parskip}
}{% else
\setlength{\parindent}{0pt}
\setlength{\parskip}{6pt plus 2pt minus 1pt}
}
\setlength{\emergencystretch}{3em}  % prevent overfull lines
\providecommand{\tightlist}{%
  \setlength{\itemsep}{0pt}\setlength{\parskip}{0pt}}
\setcounter{secnumdepth}{0}
% Redefines (sub)paragraphs to behave more like sections
\ifx\paragraph\undefined\else
\let\oldparagraph\paragraph
\renewcommand{\paragraph}[1]{\oldparagraph{#1}\mbox{}}
\fi
\ifx\subparagraph\undefined\else
\let\oldsubparagraph\subparagraph
\renewcommand{\subparagraph}[1]{\oldsubparagraph{#1}\mbox{}}
\fi

%%% Use protect on footnotes to avoid problems with footnotes in titles
\let\rmarkdownfootnote\footnote%
\def\footnote{\protect\rmarkdownfootnote}

%%% Change title format to be more compact
\usepackage{titling}

% Create subtitle command for use in maketitle
\providecommand{\subtitle}[1]{
  \posttitle{
    \begin{center}\large#1\end{center}
    }
}

\setlength{\droptitle}{-2em}

  \title{R Notebook}
    \pretitle{\vspace{\droptitle}\centering\huge}
  \posttitle{\par}
    \author{}
    \preauthor{}\postauthor{}
    \date{}
    \predate{}\postdate{}
  

\begin{document}
\maketitle

\hypertarget{exercise-4}{%
\section{Exercise 4}\label{exercise-4}}

\textbf{(1)} Test for autocorrelation on the series of interest:
\textgreater{} \textbf{(a)} Autocorrelation function (ACF)

\begin{quote}
\textbf{(b)} Partial autocorrelation function (PACF)
\end{quote}

\textbf{(2)} Test for stationarity of the process

\textbf{(3)} Find the optimal p and q for the ARMA(p,q) model:

\textbf{(4)} Run a model diagnostic on the residuals of the ARMA(p,q):
\textgreater{} \textbf{(a)} Autocorrelation function (ACF)

\begin{quote}
\textbf{(b)} Partial autocorrelation function (PACF)
\end{quote}

\begin{quote}
\textbf{(c)} Serial correlation tests: \textgreater{} \textbf{(i)}
Breusch-Godfrey test
\end{quote}

\begin{quote}
\begin{quote}
\textbf{(ii)} ARCH test
\end{quote}
\end{quote}

\textbf{(5)} Run forecasts and evaluate the results: \textgreater{}
\textbf{(a)} 1-step ahead and evaluate using RMSE and MAE

\begin{quote}
\textbf{(b)} 2-steps ahead and evaluate using RMSE and MAE
\end{quote}

\begin{Shaded}
\begin{Highlighting}[]
\FunctionTok{rm}\NormalTok{(}\AttributeTok{list=}\FunctionTok{ls}\NormalTok{())}
\FunctionTok{library}\NormalTok{(readr)}
\FunctionTok{library}\NormalTok{(here)}
\end{Highlighting}
\end{Shaded}

\begin{verbatim}
## Warning: package 'here' was built under R version 3.6.3
\end{verbatim}

\begin{verbatim}
## here() starts at C:/Users/Daniel/Desktop/Daniel/codes/python_R/FGV_Forecasing
\end{verbatim}

\begin{Shaded}
\begin{Highlighting}[]
\FunctionTok{library}\NormalTok{(aTSA)}
\end{Highlighting}
\end{Shaded}

\begin{verbatim}
## 
## Attaching package: 'aTSA'
\end{verbatim}

\begin{verbatim}
## The following object is masked from 'package:graphics':
## 
##     identify
\end{verbatim}

\begin{Shaded}
\begin{Highlighting}[]
\FunctionTok{library}\NormalTok{(data.table)}
\FunctionTok{library}\NormalTok{(xtable)}
\end{Highlighting}
\end{Shaded}

\begin{verbatim}
## Warning: package 'xtable' was built under R version 3.6.3
\end{verbatim}

\begin{Shaded}
\begin{Highlighting}[]
\FunctionTok{library}\NormalTok{(dplyr)}
\end{Highlighting}
\end{Shaded}

\begin{verbatim}
## 
## Attaching package: 'dplyr'
\end{verbatim}

\begin{verbatim}
## The following objects are masked from 'package:data.table':
## 
##     between, first, last
\end{verbatim}

\begin{verbatim}
## The following objects are masked from 'package:stats':
## 
##     filter, lag
\end{verbatim}

\begin{verbatim}
## The following objects are masked from 'package:base':
## 
##     intersect, setdiff, setequal, union
\end{verbatim}

\begin{Shaded}
\begin{Highlighting}[]
\FunctionTok{library}\NormalTok{(Hmisc)}
\end{Highlighting}
\end{Shaded}

\begin{verbatim}
## Warning: package 'Hmisc' was built under R version 3.6.3
\end{verbatim}

\begin{verbatim}
## Loading required package: lattice
\end{verbatim}

\begin{verbatim}
## Loading required package: survival
\end{verbatim}

\begin{verbatim}
## Warning: package 'survival' was built under R version 3.6.3
\end{verbatim}

\begin{verbatim}
## Loading required package: Formula
\end{verbatim}

\begin{verbatim}
## Warning: package 'Formula' was built under R version 3.6.3
\end{verbatim}

\begin{verbatim}
## Loading required package: ggplot2
\end{verbatim}

\begin{verbatim}
## Registered S3 methods overwritten by 'ggplot2':
##   method         from 
##   [.quosures     rlang
##   c.quosures     rlang
##   print.quosures rlang
\end{verbatim}

\begin{verbatim}
## 
## Attaching package: 'Hmisc'
\end{verbatim}

\begin{verbatim}
## The following objects are masked from 'package:dplyr':
## 
##     src, summarize
\end{verbatim}

\begin{verbatim}
## The following objects are masked from 'package:xtable':
## 
##     label, label<-
\end{verbatim}

\begin{verbatim}
## The following objects are masked from 'package:base':
## 
##     format.pval, units
\end{verbatim}

\begin{Shaded}
\begin{Highlighting}[]
\FunctionTok{library}\NormalTok{(lmtest)}
\end{Highlighting}
\end{Shaded}

\begin{verbatim}
## Warning: package 'lmtest' was built under R version 3.6.3
\end{verbatim}

\begin{verbatim}
## Loading required package: zoo
\end{verbatim}

\begin{verbatim}
## 
## Attaching package: 'zoo'
\end{verbatim}

\begin{verbatim}
## The following objects are masked from 'package:base':
## 
##     as.Date, as.Date.numeric
\end{verbatim}

\begin{Shaded}
\begin{Highlighting}[]
\FunctionTok{library}\NormalTok{(FinTS)}
\end{Highlighting}
\end{Shaded}

\begin{verbatim}
## Warning: package 'FinTS' was built under R version 3.6.3
\end{verbatim}

\begin{Shaded}
\begin{Highlighting}[]
\FunctionTok{library}\NormalTok{(Hmisc)}

\NormalTok{simu\_df }\OtherTok{=} \FunctionTok{read\_delim}\NormalTok{(}\FunctionTok{here}\NormalTok{(}\StringTok{\textquotesingle{}src\textquotesingle{}}\NormalTok{, }\StringTok{\textquotesingle{}data\textquotesingle{}}\NormalTok{, }\StringTok{\textquotesingle{}simu\_data\_lecture5.csv\textquotesingle{}}\NormalTok{), }\StringTok{";"}\NormalTok{, }\AttributeTok{escape\_double =} \ConstantTok{FALSE}\NormalTok{, }\AttributeTok{trim\_ws =} \ConstantTok{TRUE}\NormalTok{)}
\end{Highlighting}
\end{Shaded}

\begin{verbatim}
## Warning: Missing column names filled in: 'X1' [1]
\end{verbatim}

\begin{verbatim}
## Parsed with column specification:
## cols(
##   X1 = col_double(),
##   y1 = col_double(),
##   y2 = col_double(),
##   y3 = col_double()
## )
\end{verbatim}

\begin{Shaded}
\begin{Highlighting}[]
\CommentTok{\#head(simu\_df)}
\NormalTok{y1 }\OtherTok{=}\NormalTok{ simu\_df}\SpecialCharTok{$}\NormalTok{y1}
\end{Highlighting}
\end{Shaded}

\hypertarget{test-for-autocorrelation-on-the-series-of-interest}{%
\subsection{\texorpdfstring{\textbf{(1)} Test for autocorrelation on the
series of
interest:}{(1) Test for autocorrelation on the series of interest:}}\label{test-for-autocorrelation-on-the-series-of-interest}}

Assuming a stochastic process of the form \(\{y_t\}_{t=1}^{T}\), the
autocorrelation function is simply given by:

\$ Cov(y\_t, y\_\{t-k\}) = \gamma(k) \$

Therefore, the autocorrelation function (ACF) is defined as

\$ AC(k) = \frac{\gamma(k)}{\gamma(0)} =
\frac{Cov(y_t, y_{t-k})}{\sqrt{Var(y_t)}\sqrt{Var(y_{t-k})}} \$

and the partial autocorrelation function (PACF) is defined as the
correlation between \(y_t\) and \(y_{t-k}\) conditional on
\(Y_{-t, -(t-k)}\), which is the set of all available \(y_i\) except for
\(y_t\) and \(y_{t-k}\). This means that we can estimate the PACF using
the LRM:

\$ y\_t = \rho\emph{k y}\{t-k\} + \epsilon\_t \quad where \quad PAC(k) =
\rho\_k \$

\hypertarget{a-autocorrelation-function-acf}{%
\subsubsection{\texorpdfstring{\textbf{(a)} Autocorrelation function
(ACF)}{(a) Autocorrelation function (ACF)}}\label{a-autocorrelation-function-acf}}

\hypertarget{section}{%
\subsubsection{+}\label{section}}

\hypertarget{b-partial-autocorrelation-function-pacf}{%
\subsubsection{\texorpdfstring{\textbf{(b)} Partial autocorrelation
function
(PACF)}{(b) Partial autocorrelation function (PACF)}}\label{b-partial-autocorrelation-function-pacf}}

\begin{Shaded}
\begin{Highlighting}[]
\FunctionTok{par}\NormalTok{(}\AttributeTok{mfrow=}\FunctionTok{c}\NormalTok{(}\DecValTok{2}\NormalTok{,}\DecValTok{1}\NormalTok{))}
\NormalTok{acf.inven }\OtherTok{\textless{}{-}} \FunctionTok{list}\NormalTok{(}\StringTok{\textquotesingle{}acf\textquotesingle{}} \OtherTok{=} \FunctionTok{acf}\NormalTok{(y1, }\AttributeTok{lag.max =} \DecValTok{15}\NormalTok{, }\AttributeTok{main=}\StringTok{" ACF for y1"}\NormalTok{),}
                  \StringTok{\textquotesingle{}pacf\textquotesingle{}} \OtherTok{=} \FunctionTok{pacf}\NormalTok{(y1, }\AttributeTok{lag.max =} \DecValTok{15}\NormalTok{, }\AttributeTok{main=}\StringTok{" PACF for y1"}\NormalTok{))}
\end{Highlighting}
\end{Shaded}

\includegraphics{Time_series_models_entrega_files/figure-latex/unnamed-chunk-2-1.pdf}

Note that the AC function decays almost exponentially, while the PAC
function has two peaks and then vanishes. This present evidence that our
process is probably an AR(2) one.

\hypertarget{test-for-stationarity-of-the-process}{%
\subsection{\texorpdfstring{\textbf{(2)} Test for stationarity of the
process}{(2) Test for stationarity of the process}}\label{test-for-stationarity-of-the-process}}

\hypertarget{dickey-fuller-test}{%
\paragraph{Dickey-Fuller test}\label{dickey-fuller-test}}

To illustrate the Dickey-Fuller test, lets consider the following model:

\$ y\_t = T\_t + z\_t\textbackslash{} T\_t = \nu\_0 + \nu\emph{1
t\textbackslash{} z\_t = \rho z}\{t-1\} + \epsilon\_t \quad \epsilon\_t
\sim WN(0, \sigma\^{}2) \$

where \(T_t\) is a deterministic linear trend. We have that, if
\(\rho < 1\), then \(y_t\) is \(I(0)\) about the deterministic trend,
whereas if \(\rho=1\) and \(\nu_1=0\), then \(z_t\) is a random walk and
\(y_t\) in \(I(1)\) with drift.

Suppose \(\nu_0=\nu_1=0 \implies T_t=0\), and therefore:

\$ y\_t = z\_t = \rho z\_\{t-1\} + \epsilon\_t \quad (**) \$

We are interested in testing for unit root processes, that is, if
\(\rho=1\), which implies that \(y_t\) in \(I(1)\). We can define the
hypotheses of intereset in the following way:

\$ H\_0: \rho=1\textbackslash{} H\_1: \textbar{}\rho\textbar\textless1
\$

The problem is that, to construct the test statistics to test the above
hypotheses we would need the sample moments of \(y_t\), which under the
null is a unit root process, but unfortunatly they do not converge to
fixed constants. Dickey and Fuller (1979) derived statistics that
converge in distribution to the sample moments of \(y_t\) under the
alternative, while Phillips (1978) derived sample statistics that also
converge in probability for the null. The limiting distribution for the
t-test(\(\rho=1\)) is called the Dickey-Fueller distribution and it does
not have a closed form representation.

Its important to note that the DF distributuon is sensitive to the form
of the deterministic component. There are two most common
representations for this deterministic component:

\textbf{(1)} Constant only: The test model is given by:

\$ \Delta y\_t = \nu\emph{0 + (\rho-1) y}\{t-1\} + \epsilon\_t \$

with the following hypotheses:

\$ H\_0: \rho=1 \quad \nu\_0=0\textbackslash{} H\_1:
\textbar{}\rho\textbar\textless1 \quad \nu\_0 \neq 0 \$

\textbf{(2)} Constant and time trend: The test model is given by:

\$ \Delta y\_t = \nu\_0 + \nu\emph{1 t + (\rho-1) y}\{t-1\} +
\epsilon\_t \$

with the following hypotheses:

\$ H\_0: \rho=1 \quad \nu\_1=0\textbackslash{} H\_1:
\textbar{}\rho\textbar\textless1 \quad \nu\_1 \neq 0 \$

Furthermore, the Augmented Dickey-Fuller test expands the above model
(**) to account for more autoregressive terms.

\begin{Shaded}
\begin{Highlighting}[]
\FunctionTok{adf.test}\NormalTok{(y1)}
\end{Highlighting}
\end{Shaded}

\begin{verbatim}
## Augmented Dickey-Fuller Test 
## alternative: stationary 
##  
## Type 1: no drift no trend 
##      lag    ADF p.value
## [1,]   0 -17.58    0.01
## [2,]   1 -10.75    0.01
## [3,]   2  -9.99    0.01
## [4,]   3  -8.99    0.01
## [5,]   4  -8.40    0.01
## [6,]   5  -7.89    0.01
## Type 2: with drift no trend 
##      lag    ADF p.value
## [1,]   0 -17.62    0.01
## [2,]   1 -10.78    0.01
## [3,]   2 -10.01    0.01
## [4,]   3  -9.01    0.01
## [5,]   4  -8.42    0.01
## [6,]   5  -7.91    0.01
## Type 3: with drift and trend 
##      lag    ADF p.value
## [1,]   0 -17.74    0.01
## [2,]   1 -10.87    0.01
## [3,]   2 -10.14    0.01
## [4,]   3  -9.15    0.01
## [5,]   4  -8.56    0.01
## [6,]   5  -8.08    0.01
## ---- 
## Note: in fact, p.value = 0.01 means p.value <= 0.01
\end{verbatim}

The ADF test reject the null hypothesis of unit root process for all the
possible specifications. Therefore, we have evidence that our process
should be stationary.

\hypertarget{find-the-optimal-p-and-q-for-the-armapq-model}{%
\subsection{\texorpdfstring{\textbf{(3)} Find the optimal p and q for
the ARMA(p,q)
model:}{(3) Find the optimal p and q for the ARMA(p,q) model:}}\label{find-the-optimal-p-and-q-for-the-armapq-model}}

Recall that the AIC and BIC are measures of model fitness defined as:

\$ AIC = 2k - 2\log(\hat{L})\textbackslash{} BIC =
k\log(T)-2\log(\hat{L}) \$

where \(\hat{L}\) is the maximum value that the model likelihood
function achieves, \(k\) is the number of variables in the model, and
\(T\) is the number of observations.

We can use both AIC and BIC to find the best p and q parameters for the
ARMA(p,q) model.

\begin{Shaded}
\begin{Highlighting}[]
\NormalTok{ic.inven }\OtherTok{\textless{}{-}} \FunctionTok{list}\NormalTok{(}\StringTok{\textquotesingle{}AIC\textquotesingle{}} \OtherTok{=} \FunctionTok{data.table}\NormalTok{(), }\StringTok{\textquotesingle{}BIC\textquotesingle{}} \OtherTok{=} \FunctionTok{data.table}\NormalTok{())}
\ControlFlowTok{for}\NormalTok{ (ar.lag }\ControlFlowTok{in} \DecValTok{0}\SpecialCharTok{:}\DecValTok{11}\NormalTok{) \{}
\NormalTok{  arma.stat }\OtherTok{\textless{}{-}} \FunctionTok{rep}\NormalTok{(}\DecValTok{0}\NormalTok{, }\DecValTok{6}\NormalTok{)}
  \ControlFlowTok{for}\NormalTok{ (ma.lag }\ControlFlowTok{in} \DecValTok{0}\SpecialCharTok{:}\DecValTok{2}\NormalTok{) \{}
\NormalTok{    arma.fit }\OtherTok{\textless{}{-}} \FunctionTok{arima}\NormalTok{(y1, }\AttributeTok{order =} \FunctionTok{c}\NormalTok{(ar.lag, }\DecValTok{0}\NormalTok{, ma.lag))}
    \CommentTok{\# arma.fit}
    \CommentTok{\# AIC}
\NormalTok{    arma.stat[ma.lag }\SpecialCharTok{+} \DecValTok{1}\NormalTok{] }\OtherTok{\textless{}{-}}\NormalTok{ arma.fit}\SpecialCharTok{$}\NormalTok{aic}
    \CommentTok{\# BIC}
\NormalTok{    arma.stat[ma.lag }\SpecialCharTok{+} \DecValTok{4}\NormalTok{] }\OtherTok{\textless{}{-}} \SpecialCharTok{{-}}\DecValTok{2} \SpecialCharTok{*}\NormalTok{ arma.fit}\SpecialCharTok{$}\NormalTok{loglik }\SpecialCharTok{+}\NormalTok{ (ar.lag }\SpecialCharTok{+}\NormalTok{ ma.lag) }\SpecialCharTok{*} \FunctionTok{log}\NormalTok{(}\FunctionTok{length}\NormalTok{(y1))}
\NormalTok{  \}}
\NormalTok{  ic.inven}\SpecialCharTok{$}\NormalTok{AIC }\OtherTok{\textless{}{-}} \FunctionTok{rbindlist}\NormalTok{(}\FunctionTok{list}\NormalTok{(ic.inven}\SpecialCharTok{$}\NormalTok{AIC, }\FunctionTok{data.table}\NormalTok{(}\FunctionTok{t}\NormalTok{(arma.stat[}\DecValTok{1}\SpecialCharTok{:}\DecValTok{3}\NormalTok{]))))}
\NormalTok{  ic.inven}\SpecialCharTok{$}\NormalTok{BIC }\OtherTok{\textless{}{-}} \FunctionTok{rbindlist}\NormalTok{(}\FunctionTok{list}\NormalTok{(ic.inven}\SpecialCharTok{$}\NormalTok{BIC, }\FunctionTok{data.table}\NormalTok{(}\FunctionTok{t}\NormalTok{(arma.stat[}\DecValTok{4}\SpecialCharTok{:}\DecValTok{6}\NormalTok{]))))}
\NormalTok{\}}
\end{Highlighting}
\end{Shaded}

\begin{verbatim}
## Warning in arima(y1, order = c(ar.lag, 0, ma.lag)): possible convergence
## problem: optim gave code = 1

## Warning in arima(y1, order = c(ar.lag, 0, ma.lag)): possible convergence
## problem: optim gave code = 1

## Warning in arima(y1, order = c(ar.lag, 0, ma.lag)): possible convergence
## problem: optim gave code = 1

## Warning in arima(y1, order = c(ar.lag, 0, ma.lag)): possible convergence
## problem: optim gave code = 1
\end{verbatim}

\begin{verbatim}
## Warning in log(s2): NaNs produced
\end{verbatim}

\begin{verbatim}
## Warning in arima(y1, order = c(ar.lag, 0, ma.lag)): possible convergence
## problem: optim gave code = 1

## Warning in arima(y1, order = c(ar.lag, 0, ma.lag)): possible convergence
## problem: optim gave code = 1

## Warning in arima(y1, order = c(ar.lag, 0, ma.lag)): possible convergence
## problem: optim gave code = 1

## Warning in arima(y1, order = c(ar.lag, 0, ma.lag)): possible convergence
## problem: optim gave code = 1

## Warning in arima(y1, order = c(ar.lag, 0, ma.lag)): possible convergence
## problem: optim gave code = 1

## Warning in arima(y1, order = c(ar.lag, 0, ma.lag)): possible convergence
## problem: optim gave code = 1
\end{verbatim}

\begin{Shaded}
\begin{Highlighting}[]
\FunctionTok{setnames}\NormalTok{(ic.inven}\SpecialCharTok{$}\NormalTok{AIC, }\FunctionTok{c}\NormalTok{(}\StringTok{\textquotesingle{}MA0\textquotesingle{}}\NormalTok{, }\StringTok{\textquotesingle{}MA1\textquotesingle{}}\NormalTok{, }\StringTok{\textquotesingle{}MA2\textquotesingle{}}\NormalTok{))}
\NormalTok{ic.inven}\SpecialCharTok{$}\NormalTok{AIC[, AR }\SpecialCharTok{:}\ErrorTok{=} \DecValTok{0}\SpecialCharTok{:}\DecValTok{11}\NormalTok{]}
\FunctionTok{setnames}\NormalTok{(ic.inven}\SpecialCharTok{$}\NormalTok{BIC, }\FunctionTok{c}\NormalTok{(}\StringTok{\textquotesingle{}MA0\textquotesingle{}}\NormalTok{, }\StringTok{\textquotesingle{}MA1\textquotesingle{}}\NormalTok{, }\StringTok{\textquotesingle{}MA2\textquotesingle{}}\NormalTok{))}
\NormalTok{ic.inven}\SpecialCharTok{$}\NormalTok{BIC[, AR }\SpecialCharTok{:}\ErrorTok{=}\NormalTok{ (}\DecValTok{0}\SpecialCharTok{:}\DecValTok{11}\NormalTok{)]}


\NormalTok{BIC\_selec.mat }\OtherTok{\textless{}{-}} \FunctionTok{rbind}\NormalTok{(ic.inven}\SpecialCharTok{$}\NormalTok{BIC[, }\AttributeTok{AR :=}\NormalTok{ (}\DecValTok{0}\SpecialCharTok{:}\DecValTok{11}\NormalTok{)])}
\FunctionTok{print}\NormalTok{(}\FunctionTok{xtable}\NormalTok{(BIC\_selec.mat))}
\end{Highlighting}
\end{Shaded}

\begin{verbatim}
## % latex table generated in R 3.6.1 by xtable 1.8-4 package
## % Sun Nov 15 09:35:50 2020
## \begin{table}[ht]
## \centering
## \begin{tabular}{rrrrr}
##   \hline
##  & MA0 & MA1 & MA2 & AR \\ 
##   \hline
## 1 & 1797.47 & 1764.61 & 1715.23 &   0 \\ 
##   2 & 1740.89 & 1714.47 & 1710.97 &   1 \\ 
##   3 & 1701.19 & 1707.47 & 1713.36 &   2 \\ 
##   4 & 1707.49 & 1713.71 & 1719.76 &   3 \\ 
##   5 & 1713.42 & 1717.39 & 1723.73 &   4 \\ 
##   6 & 1719.82 & 1723.88 & 1731.81 &   5 \\ 
##   7 & 1726.16 & 1732.34 & 1734.21 &   6 \\ 
##   8 & 1731.79 & 1738.01 & 1741.21 &   7 \\ 
##   9 & 1737.86 & 1744.26 & 1747.01 &   8 \\ 
##   10 & 1744.26 & 1748.03 & 1749.79 &   9 \\ 
##   11 & 1749.69 & 1755.54 & 1759.78 &  10 \\ 
##   12 & 1754.60 & 1760.87 & 1762.39 &  11 \\ 
##    \hline
## \end{tabular}
## \end{table}
\end{verbatim}

Using both of the information criterion we see that the optimal
parameters would be ARMA(2,0).

\hypertarget{run-a-model-diagnostic-on-the-residuals-of-the-armapq}{%
\subsection{\texorpdfstring{\textbf{(4)} Run a model diagnostic on the
residuals of the
ARMA(p,q):}{(4) Run a model diagnostic on the residuals of the ARMA(p,q):}}\label{run-a-model-diagnostic-on-the-residuals-of-the-armapq}}

\hypertarget{a-autocorrelation-function-acf-1}{%
\subsubsection{\texorpdfstring{\textbf{(a)} Autocorrelation function
(ACF)}{(a) Autocorrelation function (ACF)}}\label{a-autocorrelation-function-acf-1}}

\hypertarget{section-1}{%
\subsubsection{+}\label{section-1}}

\hypertarget{b-partial-autocorrelation-function-pacf-1}{%
\subsubsection{\texorpdfstring{\textbf{(b)} Partial autocorrelation
function
(PACF)}{(b) Partial autocorrelation function (PACF)}}\label{b-partial-autocorrelation-function-pacf-1}}

\begin{Shaded}
\begin{Highlighting}[]
\NormalTok{arma\_y1 }\OtherTok{=} \FunctionTok{arima}\NormalTok{(simu\_df}\SpecialCharTok{$}\NormalTok{y1, }\AttributeTok{order =} \FunctionTok{c}\NormalTok{(}\DecValTok{2}\NormalTok{, }\DecValTok{0}\NormalTok{, }\DecValTok{0}\NormalTok{))}

\FunctionTok{par}\NormalTok{(}\AttributeTok{mfrow=}\FunctionTok{c}\NormalTok{(}\DecValTok{2}\NormalTok{,}\DecValTok{1}\NormalTok{))}
\NormalTok{acf.inven }\OtherTok{=} \FunctionTok{list}\NormalTok{(}\StringTok{\textquotesingle{}acf\textquotesingle{}} \OtherTok{=} \FunctionTok{acf}\NormalTok{(arma\_y1}\SpecialCharTok{$}\NormalTok{residuals, }\AttributeTok{lag.max =} \DecValTok{15}\NormalTok{, }\AttributeTok{main=}\StringTok{" ACF for ARMA(2,0) residuals"}\NormalTok{),}
                  \StringTok{\textquotesingle{}pacf\textquotesingle{}} \OtherTok{=} \FunctionTok{pacf}\NormalTok{(arma\_y1}\SpecialCharTok{$}\NormalTok{residuals, }\AttributeTok{lag.max =} \DecValTok{15}\NormalTok{, }\AttributeTok{main=}\StringTok{" PACF for ARMA(2,0) residuals"}\NormalTok{))}
\end{Highlighting}
\end{Shaded}

\includegraphics{Time_series_models_entrega_files/figure-latex/unnamed-chunk-8-1.pdf}

Using the correct specification of the ARMA process we can see that the
PACF has no significant terms. Furthermore, the ACF rapidly vanishes,
which is further evidence of proper model specification.

\hypertarget{c-serial-correlation-tests}{%
\subsection{\texorpdfstring{\textbf{(c)} Serial correlation
tests:}{(c) Serial correlation tests:}}\label{c-serial-correlation-tests}}

\hypertarget{breusch-godfrey-test}{%
\paragraph{Breusch-Godfrey test}\label{breusch-godfrey-test}}

\begin{Shaded}
\begin{Highlighting}[]
\NormalTok{df\_bg }\OtherTok{=} \FunctionTok{as.data.frame}\NormalTok{(}\FunctionTok{cbind}\NormalTok{(arma\_y1}\SpecialCharTok{$}\NormalTok{residuals, }\FunctionTok{Lag}\NormalTok{(arma\_y1}\SpecialCharTok{$}\NormalTok{residuals, }\DecValTok{1}\NormalTok{), }\FunctionTok{Lag}\NormalTok{(arma\_y1}\SpecialCharTok{$}\NormalTok{residuals, }\DecValTok{2}\NormalTok{)))}
\FunctionTok{colnames}\NormalTok{(df\_bg) }\OtherTok{=} \FunctionTok{c}\NormalTok{(}\StringTok{\textquotesingle{}e\textquotesingle{}}\NormalTok{, }\StringTok{\textquotesingle{}e\_l1\textquotesingle{}}\NormalTok{, }\StringTok{\textquotesingle{}e\_l2\textquotesingle{}}\NormalTok{)}
\FunctionTok{summary}\NormalTok{(}\FunctionTok{lm}\NormalTok{(e }\SpecialCharTok{\textasciitilde{}}\NormalTok{ e\_l1 }\SpecialCharTok{+}\NormalTok{ e\_l2, }\AttributeTok{data =}\NormalTok{ df\_bg))}
\end{Highlighting}
\end{Shaded}

\begin{verbatim}
## 
## Call:
## lm(formula = e ~ e_l1 + e_l2, data = df_bg)
## 
## Residuals:
##     Min      1Q  Median      3Q     Max 
## -3.8139 -0.6772  0.0554  0.6647  2.6477 
## 
## Coefficients:
##              Estimate Std. Error t value Pr(>|t|)
## (Intercept) -0.001247   0.040529  -0.031    0.975
## e_l1         0.004706   0.040962   0.115    0.909
## e_l2        -0.003081   0.041048  -0.075    0.940
## 
## Residual standard error: 0.9911 on 595 degrees of freedom
##   (2 observations deleted due to missingness)
## Multiple R-squared:  3.155e-05,  Adjusted R-squared:  -0.00333 
## F-statistic: 0.009387 on 2 and 595 DF,  p-value: 0.9907
\end{verbatim}

For both the first and second lag of the residuals of the ARMA(2,0)
model we can raise the significance level up to .90 without rejecting
the null hypothesis of uncorrelated error terms. Therefore, we have
evidence of no serial correltaion on the ARMA model.

\hypertarget{arch-test}{%
\paragraph{ARCH test}\label{arch-test}}

\begin{Shaded}
\begin{Highlighting}[]
\FunctionTok{ArchTest}\NormalTok{(arma\_y1}\SpecialCharTok{$}\NormalTok{residuals, }\AttributeTok{lags =} \DecValTok{1}\NormalTok{)}
\end{Highlighting}
\end{Shaded}

\begin{verbatim}
## 
##  ARCH LM-test; Null hypothesis: no ARCH effects
## 
## data:  arma_y1$residuals
## Chi-squared = 0.0015185, df = 1, p-value = 0.9689
\end{verbatim}

The statistics of the ARCH test says that we can increase the
significance level up to 96\% and still not reject the null hypothesis
of no ARCH effect. Therefore, we have evidence that the residuals of our
ARMA(2,0) model follows an ARCH process.

\hypertarget{run-forecasts-and-evaluate-the-results}{%
\subsection{\texorpdfstring{\textbf{(5)} Run forecasts and evaluate the
results:}{(5) Run forecasts and evaluate the results:}}\label{run-forecasts-and-evaluate-the-results}}

\hypertarget{a-1-step-ahead-and-evaluate-using-mse}{%
\subsubsection{\texorpdfstring{\textbf{(a)} 1-step ahead and evaluate
using
MSE}{(a) 1-step ahead and evaluate using MSE}}\label{a-1-step-ahead-and-evaluate-using-mse}}

\begin{Shaded}
\begin{Highlighting}[]
\NormalTok{forecas\_simu }\OtherTok{=} \FunctionTok{list}\NormalTok{()}
\NormalTok{forecas\_simu}\SpecialCharTok{$}\NormalTok{y1}\SpecialCharTok{$}\NormalTok{dynamic }\OtherTok{=} \FunctionTok{as.numeric}\NormalTok{(}\FunctionTok{predict}\NormalTok{(arma\_y1,}
                                                  \AttributeTok{n.ahead =} \DecValTok{16}\NormalTok{)}\SpecialCharTok{$}\NormalTok{pred)}
\NormalTok{forecas\_simu}\SpecialCharTok{$}\NormalTok{y1}\SpecialCharTok{$}\NormalTok{static }\OtherTok{=} \FunctionTok{rep}\NormalTok{(}\DecValTok{0}\NormalTok{, }\DecValTok{16}\NormalTok{)}
\ControlFlowTok{for}\NormalTok{ (c }\ControlFlowTok{in} \DecValTok{1}\SpecialCharTok{:}\DecValTok{16}\NormalTok{) \{}
\NormalTok{  simu\_fit }\OtherTok{=} \FunctionTok{arima}\NormalTok{(y1[}\DecValTok{1}\SpecialCharTok{:}\NormalTok{(}\DecValTok{71} \SpecialCharTok{+}\NormalTok{ c)], }\AttributeTok{order =} \FunctionTok{c}\NormalTok{(}\DecValTok{2}\NormalTok{, }\DecValTok{0}\NormalTok{, }\DecValTok{0}\NormalTok{))}
\NormalTok{  forecas\_simu}\SpecialCharTok{$}\NormalTok{y1}\SpecialCharTok{$}\NormalTok{static[c] }\OtherTok{=} \FunctionTok{predict}\NormalTok{(simu\_fit, }\AttributeTok{n.ahead =} \DecValTok{1}\NormalTok{)}\SpecialCharTok{$}\NormalTok{pred}
\NormalTok{\}}


\NormalTok{mse\_dynamic }\OtherTok{=} \FunctionTok{mean}\NormalTok{((forecas\_simu}\SpecialCharTok{$}\NormalTok{y1}\SpecialCharTok{$}\NormalTok{dynamic }\SpecialCharTok{{-}}\NormalTok{ y1[}\DecValTok{71}\SpecialCharTok{:}\NormalTok{(}\DecValTok{70}\SpecialCharTok{+}\DecValTok{16}\NormalTok{)])}\SpecialCharTok{\^{}}\DecValTok{2}\NormalTok{)}
\NormalTok{mse\_static }\OtherTok{=} \FunctionTok{mean}\NormalTok{((forecas\_simu}\SpecialCharTok{$}\NormalTok{y1}\SpecialCharTok{$}\NormalTok{static }\SpecialCharTok{{-}}\NormalTok{ y1[}\DecValTok{71}\SpecialCharTok{:}\NormalTok{(}\DecValTok{70}\SpecialCharTok{+}\DecValTok{16}\NormalTok{)])}\SpecialCharTok{\^{}}\DecValTok{2}\NormalTok{)}
\FunctionTok{print}\NormalTok{(}\FunctionTok{paste0}\NormalTok{(}\StringTok{\textquotesingle{}MSE dynamic: \textquotesingle{}}\NormalTok{, mse\_dynamic, }\StringTok{\textquotesingle{} MSE static: \textquotesingle{}}\NormalTok{, mse\_static))}
\end{Highlighting}
\end{Shaded}

\begin{verbatim}
## [1] "MSE dynamic: 0.803096864172853 MSE static: 0.673256848340654"
\end{verbatim}

\hypertarget{b-2-steps-ahead-and-evaluate-using-mse}{%
\subsubsection{\texorpdfstring{\textbf{(b)} 2-steps ahead and evaluate
using
MSE}{(b) 2-steps ahead and evaluate using MSE}}\label{b-2-steps-ahead-and-evaluate-using-mse}}

\begin{Shaded}
\begin{Highlighting}[]
\NormalTok{forecas\_simu}\SpecialCharTok{$}\NormalTok{y1}\SpecialCharTok{$}\NormalTok{multi }\OtherTok{=} \FunctionTok{rep}\NormalTok{(}\DecValTok{0}\NormalTok{, }\DecValTok{16}\NormalTok{)}
\ControlFlowTok{for}\NormalTok{ (c }\ControlFlowTok{in} \DecValTok{1}\SpecialCharTok{:}\DecValTok{16}\NormalTok{) \{}
\NormalTok{  simu\_fit }\OtherTok{=} \FunctionTok{arima}\NormalTok{(y1[}\DecValTok{1}\SpecialCharTok{:}\NormalTok{(}\DecValTok{70} \SpecialCharTok{+}\NormalTok{ c)], }\AttributeTok{order =} \FunctionTok{c}\NormalTok{(}\DecValTok{2}\NormalTok{, }\DecValTok{0}\NormalTok{, }\DecValTok{0}\NormalTok{))}
\NormalTok{  forecas\_simu}\SpecialCharTok{$}\NormalTok{y1}\SpecialCharTok{$}\NormalTok{multi[c] }\OtherTok{=} \FunctionTok{predict}\NormalTok{(simu\_fit, }\AttributeTok{n.ahead =} \DecValTok{2}\NormalTok{)}\SpecialCharTok{$}\NormalTok{pred[}\DecValTok{2}\NormalTok{]}
\NormalTok{\}}

\NormalTok{mse\_multi }\OtherTok{=} \FunctionTok{mean}\NormalTok{((forecas\_simu}\SpecialCharTok{$}\NormalTok{y1}\SpecialCharTok{$}\NormalTok{multi }\SpecialCharTok{{-}}\NormalTok{ y1[}\DecValTok{71}\SpecialCharTok{:}\NormalTok{(}\DecValTok{70}\SpecialCharTok{+}\DecValTok{16}\NormalTok{)])}\SpecialCharTok{\^{}}\DecValTok{2}\NormalTok{)}
\FunctionTok{print}\NormalTok{(}\FunctionTok{paste0}\NormalTok{(}\StringTok{\textquotesingle{}MSE multi step (2): \textquotesingle{}}\NormalTok{, mse\_multi))}
\end{Highlighting}
\end{Shaded}

\begin{verbatim}
## [1] "MSE multi step (2): 0.643141730155509"
\end{verbatim}


\end{document}
